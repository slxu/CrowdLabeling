\section{Evaluation}
Our goal is to evaluate whether our visualization toolkit is helpful for non-experts of NLP to annotate clustering and parsing training data.

\subsection{Setup}
For clustering, we use co-reference as our running example; for parsing, we ask workers to build syntactic tree for sentences according to their English skills. They don't need to tag any of the nodes. We designed the user studies as follows:
\bi
\item Participants: We asked 6 graduate students aged 21 to 26 years old to participant in our user study. 3 are native English speakers, the other 3 learned English as a second language. None of them are NLP experts. We divide them into native group and foreign group.
\item Instructions: each participant is given two answer examples one on parsing and one on clustering. And given 5 minute to learn the example and ask whatever question they have.
\item Clustering: each of them are given 10 articles for co-reference with 5 using our visualization tool and 5 conducted using Excel in the traditional way (where they will cluster expressions by putting them into the same row). The orders of the articles and the tools are random.
\item Parsing: each of them are given 10 sentences for tree parsing with 5 using our visualization tool and 5 using text editor to put parenthesis in the traditional way. For example {\em (My dog) (also likes) (eating sausage).)}
\ei
The order of the articles, the sentences and the toolkit they use are randomly shuffled.

We compare the time efficiency and accuracy to generate training data with and without the toolkit. For clustering, time efficiency is defined as the seconds worker used for each correct co-reference cluster. Accuracy is computed by purity. Let $N_{ij}$ be the number of mentions in cluster $i$ that belong to entity $j$, and let $N_j=\sum_i N_{ij}$. Then the purity of a cluster is $p_i=\max_j p_{ij}$. The overall purity is $\sum_i \frac{N_i}{N} p_i$. Purity range between 0 (bad) and 1 (good). Table~\ref{t:cluster} shows the average time, purity for 5 workers. It is clear that using our toolkit dramatically improve the efficiency and accuracy. Purity ranges between 0 (bad) and 1 (good). Table~\ref{t:cluster} shows the average time, purity for 5 workers. It is clear that using our toolkit dramatically improve the efficiency and accuracy.

\begin{table*}[bt]
\begin{center}
\begin{tabular}{|c|c|c|c|c|c|c|c|}
 \hline
 & Participant & W1 & W2 & W3 & W4 & W5 & W6 \\\hline
 & efficiency & 7.8 & 6.2 & 6.6 & 9.4 & 10.3 & 11.0\\
Text Editor& purity & 0.82 & 0.76 & 0.88 & 0.61  & 0.78 & 0.82\\\hline
& time & 5.6 & 4.5 & 6.0 & 9.5 & 8.8 & 9.1\\
Visual. Tool& purity & 0.89 & 0.86 & 0.91 & 0.72 & 0.87 & 0.83\\\hline
\end{tabular}
\end{center}
\caption{For clustering, comparing time, purity and rand index with and without the visualization toolkit} \label{t:cluster}
\end{table*}

For parsing, time efficiency is defined as the seconds worker used for each token, to compensate the time cost over long sentences.
\begin{table*}[bt]
\begin{center}
\begin{tabular}{|c|c|c|c|c|c|c|c|}
 \hline
 & Participant & W1 & W2 & W3 & W4 & W5 & W6 \\\hline
parenthesis & efficiency & 7.8 & 5.9  & 6.7 & 4.7 & 6.6 & 8.9\\\hline
Visual. Tool& time & 4.3 & 4.8 & 4.4& 3.7 & 5.4 & 6.7\\\hline
\end{tabular}
\end{center}
\caption{For parsing, compare time with and without the visualization toolkit} \label{t:cluster}
\end{table*}
%We manually check the quality of these tree and find out that the qualities with and without the toolkit is quite comparable. It is reasonable because we provide workers text edits with the function of parenthesis matching, so people can easily figure out the errors of parenthesis in the sentences. It shows that people are more intolerable about the errors in parsing, even with text editors. But it is clear that they spent much more time on each tree. During crowd-sourcing, time cost equals money spending. It shows that our toolkit is still valuable.

Our clustering tool achieved an average 1.0s/15.2\% decrease in time consumption and at the same time an average 6.6\% increase of purity on clustering result. Our tree parsing tool achieved an average of 1.9s/27.8\% decrease in time consumption for parsing. We manually checked the quality of these tree and find out that the qualities with and without the toolkit is quite comparable. It is reasonable because we provide workers text edits with the function of parenthesis matching, so people can easily figure out the errors of parenthesis in the sentences. It shows that people are more intolerable about the errors in parsing, even with text editors. But it is clear that they spent much more time on each tree. During crowd-sourcing, time cost equals money spending. It shows that our toolkit is still valuable.



