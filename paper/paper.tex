% 8 pages + 2 pages of refs

\documentclass[11pt]{article}

%\usepackage{naaclhlt2013}
\usepackage{acl2013}
\usepackage{graphicx}
\usepackage{algorithmic}
\usepackage{times}
\usepackage{latexsym}
\usepackage{multirow}
\usepackage{url}
\usepackage{subfigure}
\usepackage{array}
\usepackage{sparklines}
\usepackage{amsmath,amsfonts,amssymb,amsthm}



\usepackage{paralist}
\usepackage{color}
%\usepackage{setspace}



\renewcommand{\baselinestretch}{0.98}
%\let\env=\texttt
%\let\pkg=\textsf
%\definecolor{sparkrectanglecolor}{rgb}{0.8,0.95,0.8}
%\definecolor{sparkspikecolor}{rgb}{1,0,0}
%\setlength{\sparkdotwidth}{1.3pt}
%\setlength{\sparklinethickness}{.3pt}

\DeclareMathOperator*{\argmax}{arg\,max}
  %\setlength\titlebox{6.5cm}    % Expanding the titlebox

\newcommand{\comment}[1]{} 
\newcommand{\NOTE}[1]{\marginpar{\em {#1}}}
\newcommand{\bug}
    {\mbox{\rule{2mm}{2mm}}}
\newcommand{\Bug}[1]
    {\bug \footnote{BUG: {#1}}}
\newcommand{\QED}{\mbox{$\Box$}}
\newcommand{\ddt}{\mbox{$\frac{d}{dt}\,$}}  %* no strut

\newcommand{\etal}{\mbox{\it et al.}}
\newcommand{\etc}{\mbox{\it etc.}}
\newcommand{\eg}{\mbox{\it e.g.}}
\newcommand{\ie}{\mbox{\it i.e.}}
\newcommand{\cf}{\mbox{\it cf.}}
\newcommand{\Eg}{\mbox{\it E.g.}}
\newcommand{\Ie}{\mbox{\it I.e.}}
\newcommand{\?}{\mbox{?}}




\newcommand{\DEFN}{\mbox{$\stackrel{\mbox{\tiny def}}{=}$}}

\newcommand{\TT}[1]{\mbox{\tt #1}}
%\newcommand{\bi}{\begin{itemize}}
%\newcommand{\ei}{\end{itemize}}
\newcommand{\bi}{\begin{list}{$\bullet$}{
    \setlength{\leftmargin}{1.5 em}
    \setlength{\itemsep}{0 pt}
    \setlength{\topsep}{3 pt}
    \setlength{\parsep}{3 pt}
    \setlength{\partopsep}{0 pt}
    \setlength{\labelwidth}{1 em}
    \setlength{\labelsep}{0.5 em}
    \setlength{\parskip}{0cm}  }}
\newcommand{\ei}{\end{list}}

\newcommand{\BE}{\begin{enumerate}}
\newcommand{\EE}{\end{enumerate}}

%\newtheorem{defnctr}{Definition}
%\newtheorem{theorem}{Theorem}
%\newtheorem{lemma}[theorem]{Lemma}
%\newtheorem{corollary}[theorem]{Corollary}        
%\newtheorem{conjecture}[theorem]{Conjecture}
%\newtheorem{propos}[theorem]{Proposition}
%\newcommand{\proof}
%	{\vspace{-8pt}
%	 {\bf Proof:}}



\newcommand{\tuple}[1]
        {\mbox{$\langle{#1}\rangle$}}
\newcommand{\set}[1]
        {\mbox{$\{{#1}\}$}}
\newcommand{\size}[1]{\mbox{$\mid\!#1\!\mid$}}



\newcommand{\PROOF}[1]{\mbox{\noindent \proofbegin {#1} \proofend}}
\newcommand{\proofbegin}{\mbox{\bf Proof: \ }}
\newcommand{\proofend}{\Math{\Box}}
\newcommand{\st}{\mbox{ such that }}
\newcommand{\wht}{\mbox{ we have that }}
\newcommand{\entails}{\mbox{$\models$}}
\newcommand{\yields}{\mbox{$\models$}}
\newcommand{\dgets}{\mbox{$\leftarrow$}}

%\renewcommand{\baselinestretch}{0.985}




\newcommand{\initab}{                           % set up tab stops
\begin{tabbing}
XXX \= XXXX \= \kill
}
\newcommand{\begpub}{
\begin{quotation}
\noindent
}
\newcommand{\nextpub}{

\vspace{2mm}
\noindent
}
\newcommand{\finpub}{
\end{quotation}
}



\hyphenation{non-de-ter-mi-nis-tic-al-ly non-de-ter-mi-nis-tic
exis-ten-tial-ly quan-tified se-lec-tion exis-ting in-stan-tiated
uni-vers-al-ly es-tab-lish in-con-sis-tent}


% \newcommand{\ncd}{\mbox{$\neq$}}
% \newcommand{\cd}{\mbox{$=$}}
% \newcommand{\fabian}{\mbox{\sc fabian}}
% \newcommand{\occam}{\mbox{\sc occam}}
% \newcommand{\sadl}{\mbox{\sc sadl}}
% \newcommand{\socrates}{\mbox{\sc socrates}}
% \newcommand{\uwl}{\mbox{\sc uwl}}
% \newcommand{\spa}{\mbox{\sc spa}}
% \newcommand{\scr}{\mbox{\sc scr}}
% \newcommand{\strips}{\mbox{\sc strips}}
% \newcommand{\snlp}{\mbox{\sc snlp}}
% \newcommand{\cbur}{\mbox{\sc C-buridan}}
% \newcommand{\buridan}{\mbox{\sc buridan}}
% \newcommand{\xii}{\mbox{\sc xii}}
% \newcommand{\zeno}{\mbox{\sc zeno}}
% \newcommand{\adl}{\mbox{\sc adl}}
% \newcommand{\kr}{\mbox{\tt /kr94}}
% \newcommand{\ucpop}{\mbox{\sc ucpop}}
% 
% \newcommand{\cause}{{\tt cause}}
% \newcommand{\observe}{{\tt observe}}
% \newcommand{\satisfy}{{\tt satisfy}}
% \newcommand{\handsoff}{{\tt hands-off}}
% \newcommand{\findout}{{\tt find-out}}
% 
% \newcommand{\CAUS}{\mbox{C}}
% \newcommand{\OBS}{\mbox{O}}
% \newcommand{\PAT}{\mbox{\sf E}}
% \newcommand{\fact}{\mbox{$\varphi$}}
% \newcommand{\LIT}{\mbox{\sf p}}
% \newcommand{\lit}{\mbox{\LIT}}
% \newcommand{\litp}{\mbox{\LIT$^\prime$}}
% \newcommand{\rel}{\mbox{REL}}
% \newcommand{\change}[3]{\mbox{$\Delta(#1,{\tt #2}\rightarrow {\tt #3})$}}   
% 
% \newcommand{\true}{{\tt T}}
% \newcommand{\false}{{\tt F}}
% \newcommand{\unknown}{{\tt U}}
% 
% 


\def\upcase{\expandafter\makeupcase}



\newcommand{\hrone}{temporal functionality heuristic}
\newcommand{\hrtwo}{temporal burstiness heuristic}
\newcommand{\hrthree}{one event-mention per discourse heuristic}
\newcommand{\hrfour}{Tense Consistency Heuristic}


\newcommand{\temporal}{Temporal Hypotheses}
\newcommand{\sys}{\mbox{\sc CrowdAnno}}   % afar
\newcommand{\kylin}{\mbox{\sc Kylin}}
\newcommand{\tr}{\mbox{\sc TextRunner}}
\newcommand{\E}{\mbox{$\mathbf e$}}

%\newcommand{\mtt}[1]{\mbox{$\tt{#1}$}}
\newcommand{\bmtt}[1]{\mbox{$\tt{\mathbf{#1}}$}}

\newcommand{\eat}[1]{}

\newcommand{\Eec}{\mbox{\em extracted event candidate}}
%\newcommand{\Eec}{Extracted Event Candidate}
%\newcommand{\Eec}{\mbox{Extracted Event Candidate}}
\newcommand{\eec}{\mbox{EEC}}
\newcommand{\bag}{\mbox{EEC-set}}

\newcommand{\mtt}[1]{{\it #1}}


\title{Visualizing NLP annotations for Crowdsourcing }

%\title{Distant Supervision for Information Extraction of Polymorphic Tuples}
%\title{Distant Supervision for Information Extraction
%                   without Relational Exlcusivity}


\author{Hanchuan Li, Haichen Shen, Shengliang Xu and Congle Zhang\\
 Computer Science \& Engineering \\
 University of Washington\\
 Seattle, WA 98195, USA \\
 {\tt \{hanchuan,haichen,shengliang,clzhang\}@cs.washington.edu} \\}


%\date{}




\begin{document}

\maketitle

\begin{abstract}
Visualizing NLP annotation is useful for the collection of training data for the statistical NLP approaches.  Existing toolkits either provide limited visual aid, or introduce comprehensive operators to realize sophisticated linguistic rules. Workers must be well trained to use them. Their audience thus can hardly be scaled to large amounts of non-expert crowd-sourced workers. In this paper, we present \sys, a visualization toolkit to allow crowd-sourced workers to annotate two general categories of NLP problems: clustering and parsing. Workers can finish the tasks with simplified operators in an interactive interface, and fix errors conveniently. User studies show our toolkit is very friendly to NLP non-experts, and allow them to produce high quality labels for several sophisticated problems. We release our source code and toolkit to spur future research.
\end{abstract}


\section{Introduction}

With the popularity of the Intenet, there are huge amount of text in the web, and their size is still growing quickly. In order to use them, it is of utmost importance to develop automatic nature language processing (NLP) systems to handle the big data. 

%A typical NLP system would first employ syntactic processing and then employ semantic processing.  Syntactic processing is often task independent. It aims to convert the raw text into some machine friendly structures. The pipeline often includes tokenization, POS tagging, parsing, name entity recognition and coreference. Semantic processing is often task dependent. It tries to exploit useful information from the text for an end  task. 

It has been wildly accepted that statistical machine learning approaches are very effective for most NLP problems, such as parsing, relation extractions, question answering and so on. However, there is a significant limitation of all statistical approaches, they need a lot of training data to train the model. These training data are labeled by annotators. In most cases, annotators manually predict the desired outputs for the inputs. The algorithms then learn the statistics and models from the training data and use them to automatically predict the output of the unlabeled data.   

Generally, labeling datasets for machine learning problems (often classification) is time consuming and tedious. To make matters worse, it is even harder for annotators to label NLP problems than to label standard classification problems. For a standard classification problem, human annotators are given a set of labels and a list of objects. Their jobs are to choose a label for each object. Since objects are usually independent of each other, the prediction is local and straightforward. Besides, labeling errors, if any, would not affect the rest of the dataset. Therefore, the user interface for the annotations could be very naive: plain text or Excel tables are often enough in many cases. 


But for NLP annotations, the outputs are often structured predictions. That is, each prediction is depending on some other prediction. For example, Figure 1 shows how parsing algorithm converts a sentence into a tree. It is hard for Annotates to decide the position of a single word in the tree before drawing the whole tree. Besides, a large group of NLP problems are related to clustering, such as coreference, which clusters mentions in the article of the same entities. Unlike classification, annotates must understand the big picture in order to correctly label the clusters of the data. Transitivity makes the annotation very tricky. For example, after merging pairs of points, the annotator has created two clusters $\{A_1,\ldots,A_{10}\}$ and $\{B_1,\ldots,B_{10}\}$. He then accidentally merges $A_1$ and $B_2$. If this operation is incorrect, it would immediately cause $100$ pairs of errors, which is really a disaster. 


The challenges listed above makes the labeling process extremely uncomfortable for normal annotators. Firstly, annotators must spend a lot of time to understand the inner structure of the data before labeling anything. Secondly, annotators would often revisit and edit their labels. For example, during coreference annotation, at first annotators did not know there are two ``Obama"s in the article so he simply annotate all ``Obama" as ``Barack Obama". After a while, he have to go back to fix them because he noticed ``Mitchell Obama". False annotations tend to occur during such trial and errors. In fact, even trained linguistic must spend a lot of time to label NLP datasets. For example, it costs 8 years to create the Penn Tree Bank, a labeled set of parsed trees. 

Nowadays, there are many people work at crowdsourcing platforms like Mechanical Turks and Odesk. This phenomena provides an opportunity to quickly collect a large amount of training data. But most of them are normal people, having little knowledge about linguistic, and having no reason to be patient enough. It is impractical to ask them to label over plain text files or Excel tables, since they would switch to other easier profitable tasks.
\section{Proposed System}

In this project, we aim to develop a visualized toolkit for crowdsourcing NLP annotations. The target audience are normal people with little knowledge and patience. The toolkit would allow them to quickly label NLP datasets.

There are two key properties of our toolkit: firstly, annotators could interact with the points to understand the data in an up-to-date way. For example, any partial annotations reflect annotators' partial understanding. So they would expect immediate feedback from the toolkit. Secondly, the toolkit should enable and even encourage trial and errors. It would not take any edits from the users as granted, but treat the edits as clues to better render the data to the annotator. When the annotator finish a labeling task, he should be satisfied with the global outcome and confident. In the above $\{A_1,\ldots,A_{10}\}$ and $\{B_1,\ldots,B_{10}\}$ example, it is likely that $A_1$ and $B_2$ are hard to distinguish when the pair is seen separately from the rest of the data. But if the toolkit could immediate show a big cluster $\{A_1,\ldots,A_{10}, B_1,\ldots,B_{10}\}$ after accidentally merge $A_1$ with $B_2$, the annotators would have a good chance to change their mind and fix the errors. 

 In this project, we would focus on two important kinds of NLP annotations: a tree prediction (\eg\  parsing) and a graph prediction (\eg\ coreference). But we would keep in mind that the toolkit should be easily extensible to any NLP problems. 

\subsection{Annotate Tree Prediction}
\subsection{Annotate Graph Prediction}
\section{Clustering Annotation}

Clustering task usually consists of tens or hundreds of targeted tokens that need to be clustered together. 

\subsection{View}

The entire window is separated into two parts. The left part (about 40\%) is used to display the document(s). We highlight all the targeted tokens with a grey background and white texts to make them more obvious and easier to notice. The right partition (about 60\%) is the operation area where user group related tokens together. Supporting operation will be introduced in next subsection. Each token has a corresponding node in the operation area. To bridge the connection between the text display and graphic chart, we label each token and its corresponding node with an index.

\subsection{Manipulation}

\paragraph{\textsc{Select}\\}

\paragraph{\textsc{Drag}\\}

\paragraph{\textsc{Link Add/Remove}\\}

\paragraph{\textsc{Color Scheme}\\}
%Evaluation of the system:

We will have separate evaluations of the two proposed visualization tools we will build for tow different tasks of natural language processing.

Clustering visualization
In order to evaluate the clustering visualization tool, we will compare that to a traditional linguistics  process of doing word clustering, where the participants are required to manually label each word according to different categories on a Microsoft Excel or similar chart software.

Participants: We will gather 10 native speaker participants form undergraduate/graduate CSE students.
Experiment: We will divide them into two groups, Each group will complete two standard linguistic clustering tasks with similar work load and difficulty. One of the group will conduct the task with help of our visualization tool first, the other will do the task without visualization first. Then the tow groups switch task.

Evaluation: Both time consumption and accuracy of the two groups of participants will be evaluated.

2. Tree Parsing

Because of the difficulty of the Tree Parsing tasks, there is actually no reliable way for people to conduct such tree parsing without tedious training. So our evaluation aim to discover how this good this tool can actually turn something almost impossible to reality.

Participants: We will gather 10 native speaker participants form undergraduate/graduate CSE students.
Experiment: participants will complete two standard linguistic parcing tasks with similar work load and difficulty. One of the group will conduct the task with help of our visualization tool first, the other will do the task without visualization first. Then the two groups switch task.

Evaluation: Both time consumption and accuracy of the two groups of participants will be evaluated.
\section{Related Work}

Many NLP tasks require large amount of high quality training data.
Manual annotation for such training data is well-known for its tedium.
To generate a comprehensive annotated training set requires much human
effort. Annotators are also prone to make mistakes during the long and
tedious annotating process.  Researchers are trying to address these
problems by two means: 1) building specialized annotating tools to
ease the annotating process in the hope of improving efficiency as
well as reducing the error rates; 2) adopting crowdsourcing to scale
up annotating.

\textbf{Specialized annotating tools}. Facing one of the biggest common problems, many NLP researchers have developed a number of tools
for annotating training corpora along the history of NLP research. At
  first, before the blossom of the web, tools are generally built as
  local programs such as the WordFreak linguistic annotation
  tool~\cite{Morton2003WOT} and the  UAM CorpusTool for
  text and image annotation~\cite{ODonnell2008DUC}.
  These tools are very restricted because they cannot scale. Web-based
  annotation tools are developed later in order to scale up the
  annotating process, such as~\cite{Stuhrenberg2007WAA}.
  However these tools typically only use very basic HTML based
  techniques to provide very limited visual aids for the annotating
  process. Most related in scope is~\cite{yan2012collaborative} which
  provides a collaborative tool to assist annotators in tagging of
  complex Chinese and multilingual linguistic data. It visualizes a
  tree model that represents the complex relations across different
  linguistic elements to reduce the learning curve. Besides it
  proposes a web-based collaborative annotation approach to meet the
  large amount of data.  Their tool only focuses on a specific area
  that is complex multilingual linguistic data, whereas our work is
  trying to address how to generate a visualization model for general
  data sets.


\textbf{Crowdsoursing in NLP}. Crowdsourcing \cite{howe2006rise} is a
popular and fast growing research area. There have been a lot of
studies on understanging what it is and what it can do. For instance,
\cite{quinn2009taxonomy} categorizes crowdsourcing into seven genres:
Mechanized Labor, Game with a Purpose (GWAP), Widom of Crowds,
Crowdsourcing, Dual-Purpose Work, Grand Serarch, Human-based Genetic
Algorithms and Knowledge Collection from Volunteer Contributors. Other
works, such as \cite{abekawa2010community} and \cite{irvine2010using},
develops a specific tool and verifies the feasibility and benefit of
crowdsourcing. It is generally convinced that crowdsourcing is of
great beneficial if the tasks are easy to conduct by the workers and
the tasks are independent.

Because of the high labor requirements in typical NLP training tasks,
there also have been some work considering using crowdsoursing in many
NLP tasks. For example, Grady \etal\ generated a data set on document
relevance to search queries for information
retrieval~\cite{Grady2010CDR18666961866723}; Negri \etal\ built a
cross-lingual textual corpora~\cite{Negri2011DCC21454322145510};
Finin \etal\ collected simple named entity annotations using Amazon MTurk
and Crowd-Flower~\cite{Finin2010ANE18666961866709}. Also there are
some researchers observed the hardness of collecting high quality data
and did some studies on improving that, such
as~\cite{Hsueh2009DQC15641311564137}( how annotations should be
selected to maximize quality), and \cite{lease2011quality} (quality
control in crowdsoursing by machine learning).

Different from previous studies, we seek to improve crowdsoursing
annotating quality by greatly lower the usability barrier through the
proposed visualized toolkit rather than trying to cleaning up the data
generated by the crowdsoursing process.






%The source code of our system, its output, and all data annotations are available at {\tt http://cs.uw.edu/homes/raphaelh/mr}.
%
\bibliographystyle{naaclhlt2013}
\bibliography{general,paper}

\end{document}


