\section{Project Plan} 

\subsection{System overview} In this project,
we aim to develop a visualized toolkit for crowd-sourcing NLP
annotations. The target audience are normal people with little
knowledge and patience. The toolkit would allow them to quickly label
NLP datasets.

There are two key properties of our toolkit: firstly, annotators could
interact with the data to understand them in a refresh way. Annotators
label some examples and they expect immediate feedback from the
toolkit. These feedbacks will help them understand the problem.
Secondly, the toolkit should enable and encourage trial and errors. It
would not assume any edits from the users as gold, but treat the edits
as clues to better visualize the data to the annotator. When the
annotator finishes a labeling task, he should be satisfied and
confident with the overall outcome. For example, it is hard to
distinguish whether ``Jeff" is ``Jeff Bilmes" or ``Jeffery Heer" when
data points are seen individually. But if the toolkit could immediate
show a big cluster $\{$``Jeff, Jeff Bilmes, Jeffery Heer, Professor
Heer"$\}$ after incorrectly merge two points, the annotators would
have a good chance to fix it.

\subsection{System detailed design}

We propose to build our system as a web application for collabrative
annotation because we are targeting our toolkit as deployable by
scalable crowsoursing systems. Based on this requirement, we plan to
build the toolkit based on the D3 web visualization library.

\subsubsection{Input Output design} As a collaborative web
application, the input/output data must be sharable by different
annotators. We plan to use the data storage service provided by the
Google app engine. 

\subsubsection{Data representation design} By survying a lot of
existing NLP tasks, we decide to focus on two types of annotating
data. 1) Given an article or a webpage and a list of entities
represented by words or phrases, where the entities appear in the
article, annotate the entities; 2) Given a list of sentences,
paragraphs or articles, directly annotate them. 


\subsubsection{Task visualization design}

In this project, we would focus on two important kinds of NLP
annotations: building trees (\eg\  parsing) and clustering (\eg\
coreference resolution). 

We propose to build a new D3 tree plugin for conducting the tree
building task by visualizing the in-building trees. The users can
directly operate on the visualized tree to complete the whole annoting
process.  In addition to the tree builidng from a set of unstructured
data points, we also plan to support tree evolving, i.e. building
other trees from an existing tree. This feature is applicable to many
cross-lingual tasks such as mapping a semantic tree of an English
sentence to the tree of the translated Chinese sentence.

For the clustering task, we propse to do it by building a graph based
clustering plugin on D3. The users can directly operate on the
clustering graph to finish the clustering annotating processing.

\subsection{Milestones}

\emph{System Brainstorming}
\\All group members work on this together.
\\[1\baselineskip]
\emph{System Input Output Implementation:}\\Major Responsibility: Congle Zhang \\Minor Responsibility: Shengliang Xu, Haichen Shen
\\[1\baselineskip]
\emph{System Graphic \& Visualization Implementation:} \\Major Responsibility: Haichen Shen, Shengliang Xu \\Minor Responsibility: Hanchuan Li, Congle Zhang 
\\[1\baselineskip]
\emph{System Layout Adjustment \& User Evaluation Study:} \\Major Responsibility: Hanchuan Li \\ Minor Responsibility: Congle Zhang, Haichen Shen, Shehgliang Xu.
